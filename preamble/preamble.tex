% =====================================================================
% DOKUMENTENKLASSE UND GRUNDKONFIGURATION
% =====================================================================
% scrreprt ist die KOMA-Script-Klasse für Berichte/Arbeiten (ersetzt 'report')
\documentclass[
    ngerman,        % Sprache: Neue deutsche Rechtschreibung
    12pt,           % Schriftgröße
    a4paper,        % Papierformat (oft Standard, hier explizit)
    headings=normal,% Größe der Überschriften
    parskip=half,   % Halbzeiliger Abstand zwischen Absätzen (statt Einzug)
    bibliography=totoc, % Literaturverzeichnis im Inhaltsverzeichnis anzeigen
    index=totoc,    % Index im Inhaltsverzeichnis anzeigen
    %BCOR=10mm,     % Bindekorrektur (optional, je nach Druckerei)
]{scrreprt}

\usepackage{subfiles}

% =====================================================================
% SPRACHE, KODIERUNG UND TYPOGRAFIE
% =====================================================================
\usepackage[T1]{fontenc}    % Wichtig für Umlaute und "ß" in PDFs (Suchen/Kopieren)
\usepackage{babel}          % Deutsche Sprachanpassungen (Silbentrennung, Übersetzungen)
\usepackage[autostyle=true]{csquotes} % Automatisch korrekte deutsche Anführungszeichen
\usepackage{microtype}      % Verbessert das Schriftbild durch fortgeschrittene Typografie
\usepackage{lmodern}        % Verbessert die Standard-Schriftart
\usepackage{xcolor}
% =====================================================================
% LAYOUT UND SEITENRÄNDER (optional, KOMA hat gute Standards)
% =====================================================================
% Beispiel für die Nutzung von geometry, falls Ränder angepasst werden müssen
% \usepackage[left=2.5cm, right=2.5cm, top=3cm, bottom=2.5cm]{geometry}

% =====================================================================
% MATHEMATIK UND EINHEITEN
% =====================================================================
\usepackage{amsmath}        % Erweiterte Funktionen für mathematische Formeln
\usepackage{amssymb}        % Zusätzliche mathematische Symbole (Mengenzeichen etc.)
\usepackage{siunitx}        % Für korrekt formatierte Zahlen und Einheiten (z.B. \SI{100}{\meter\per\second})
\sisetup{output-decimal-marker={,}} % Optional: Stellt den Dezimaltrenner auf Komma

% =====================================================================
% ABBILDUNGEN, TABELLEN UND CODE
% =====================================================================
\usepackage{graphicx}       % Zum Einbinden von Bildern (\includegraphics)
\usepackage{booktabs}       % Für professionell aussehende Tabellen (keine vertikalen Linien)
\usepackage{caption}        % Anpassung der Formatierung von Bild- und Tabellenunterschriften
\usepackage{subcaption}     % Für Unterabbildungen (zwei Bilder nebeneinander)

\usepackage{sourcecodepro}
\usepackage{listings}
\definecolor{py_comment}{HTML}{008000}   % Grün
\definecolor{py_keyword}{HTML}{0000FF}   % Blau
\definecolor{py_builtin}{HTML}{900090}   % Lila (für print, len, etc.)
\definecolor{py_string}{HTML}{A31515}    % Dunkelrot
\definecolor{py_number}{HTML}{098658}    % Blau-Grün
\definecolor{py_decorator}{HTML}{795E26} % Braun
\lstset{
    language=Python,
    basicstyle=\ttfamily\fontsize{9pt}{13pt}\selectfont,
    commentstyle=\color{py_comment}\itshape,
    keywordstyle=\color{py_keyword}\bfseries,
    stringstyle=\color{py_string},
    numberstyle=\tiny\color{gray},
    % Zusätzliche Python-Keywords für mehr Farbe
    emph={self, cls, @classmethod, @staticmethod},
    emphstyle=\color{py_decorator}\bfseries,
    % Eingebaute Funktionen (Built-ins) farblich absetzen
    morekeywords=[2]{print, len, range, sum, max, min, enumerate, zip, isinstance, type, list, dict, set, str, int, float},
    keywordstyle=[2]\color{py_builtin},
    % Design-Einstellungen
    numbers=left,
    numbersep=5pt,
    breaklines=true,
    showstringspaces=false,
    frame=lines,           
    framerule=1pt,
    rulecolor=\color{orange!100},
    xleftmargin=10pt,
    tabsize=4,
    aboveskip=10pt,
    belowskip=10pt
}
\captionsetup[lstlisting]{
    font={footnotesize, it},      % 'small' macht es kleiner, 'it' macht es kursiv
    labelfont={bf},
    skip=10pt,
    justification=centering % Zentriert die Beschriftung (optional)
}

% =====================================================================
% LITERATURVERZEICHNIS UND ZITATE
% =====================================================================
% Der moderne Standard: biblatex mit biber als Backend
\usepackage[
    backend=biber,
    style=authoryear,  % Zitierstil (z.B. Autor (Jahr))
    datamodel=ngerman, % Deutsche Anpassungen für biblatex
    % natbib=true,     % Optionale Kompatibilität zu natbib Befehlen
]{biblatex}
% Pfad zur BibTeX-Datei angeben
% \addbibresource{literatur.bib}

% =====================================================================
% ABKÜRZUNGSVERZEICHNIS UND GLOSSAR
% =====================================================================
\usepackage{acro} % Paket für Abkürzungsverzeichnis
% Beispieldefinition einer Abkürzung:
% \DeclareAcronym{latex}{short=LaTeX, long={Lamport's Tex}}

% =====================================================================
% VERWEISE UND LINKS (MUSS FAST ZULETZT GELADEN WERDEN!)
% =====================================================================
\usepackage{hyperref}       % Macht Links, Verweise und URLs klickbar im PDF
\hypersetup{
    colorlinks=true,        % Färbt Links
    linkcolor=black,        % Farbe für interne Links (Kapitel, Seiten)
    citecolor=black,        % Farbe für Zitate
    urlcolor=blue,          % Farbe für URLs
    pdfborder={0 0 0},      % Entfernt die hässlichen Link-Rahmen
}
\usepackage{cleveref}       % Intelligente Querverweise (fügt "Abbildung" oder "Tabelle" automatisch hinzu)
% =====================================================================
% EPIGRAPH (SCHÖNE ZITATE AN KAPITELANFÄNGEN)
% =====================================================================
\usepackage{epigraph}
% Einstellungen für ein schöneres Layout
\setlength{\epigraphwidth}{0.55\textwidth} % Breite
\renewcommand{\epigraphsize}{\small}      % Schriftgröße
\renewcommand{\sourceflush}{flushright}    % Quelle rechtsbündig

% Optional: Quelle in Kapitälchen oder Fett
\renewcommand{\textflush}{itshape} % Zitat selbst kursiv
\makeatletter
\makeatletter
\renewcommand{\@episource}[1]{%
   %\vspace{0.5\baselineskip} % Optional: Hier können Sie den Abstand manuell feintunen
  {\raggedleft \scshape #1\par} % \raggedleft statt flushright spart Platz
}
\makeatother


% =====================================================================
% NÜTZLICHE WERKZEUGE (MUSS ZULETZT GELADEN WERDEN!)
% =====================================================================
\usepackage[textsize=tiny]{todonotes} 
