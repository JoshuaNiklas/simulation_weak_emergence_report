\documentclass[../main.tex]{subfiles}
\begin{document}
\chapter{Einleitung}
\epigraph{\enquote{Das Ganze ist mehr als die Summe seiner Teile.}}{Aristoteles}
\todo{Füge Text für Einleitung hinzu}
Man nehme eine zufällig verteilte Population von Agenten $X$.
Es gilt $X = \{x_1, x_2, ..., x_N\}$ mit $N \in \mathbb{N}$ als Anzahl der Agenten.
Für einen Agenten $x_i \in X$ gilt
\begin{equation}
    x_i = Lb + r \cdot (Ub-Lb) \quad\text{mit } i = 1,2, ..., N ,
\end{equation}
wobei $Lb$ das untere und $Ub$ das obere Limit des Definitionsbereichs ist.


Für die Nachbarschaft eines Agenten $x_i$ wird der euklidische Abstand verwendet.
Mithilfe eines Radiuses $r$ wird die Nachbarschaft $\mathcal{N}_i \subseteq X$ eines Agenten $x_i$ definiert als
\begin{equation}
    \mathcal{N}_i = \{x \in X \mid d(x_i, x) \leq r \} \quad\text{mit } d(x_i, x) = |x_i - x| .
\end{equation}

Zur Ermittlung der Nachbarschaft im $n$-dimensionalen Raum wird folgender Algorithmus
verwendet:

\begin{lstlisting}[caption={Algorithmus zur Bestimmung der Nachbarschaft eines Agenten im $n$-dimensionalen Raum}, label={lst:neighborhood_algorithm}]
def get_neighborhood(agent, all_agents, radius):
    neighborhood = []
    for other_agent in all_agents:
        distance = euclidean_distance(agent.position, other_agent.position)
        if distance <= radius:
            neighborhood.append(other_agent)
    return neighborhood
\end{lstlisting}

\end{document}